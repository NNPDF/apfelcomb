\documentclass[11pt]{article}
\usepackage{geometry}                % See geometry.pdf to learn the layout options. There are lots.
\usepackage{amsmath,amsfonts,amssymb}
\usepackage{longtable}
\usepackage{graphicx}
\usepackage{subfig}
\usepackage[numbers, sort]{natbib}
\usepackage{hyperref}
\geometry{letterpaper}                   % ... or a4paper or a5paper or ... 
%\geometry{landscape}                % Activate for for rotated page geometry
%\usepackage[parfill]{parskip}    % Activate to begin paragraphs with an empty line rather than an indent
\usepackage{graphicx}
\usepackage{amssymb}
\usepackage{epstopdf}
\DeclareGraphicsRule{.tif}{png}{.png}{`convert #1 `dirname #1`/`basename #1 .tif`.png}

\newcommand{\be}{\begin{equation}}
\newcommand{\ee}{\end{equation}}
\newcommand{\ba}{\begin{eqnarray}}
\newcommand{\ea}{\end{eqnarray}}

\newcommand{\ns}{\normalsize}

%%%%%%%%%%%%%%%%%%%%%%%%%%%%%%%%%%%%%%%%%%
%%%%%%%%%%%%%%%%%%%%
% abbreviate Greek
\newcommand{\ax}{\alpha}
\newcommand{\bx}{\beta}
\newcommand{\cx}{\gamma}
\newcommand{\dx}{\delta}
\newcommand{\ox}{\omega}
\newcommand{\lx}{\lambda}
\newcommand{\gx}{\gamma}
\newcommand{\ab}{\bar\alpha}
\newcommand{\bb}{\bar\beta}
\newcommand{\cb}{\bar\gamma}
\newcommand{\db}{\bar\delta}
\newcommand{\Sx}{\Sigma}
\newcommand{\Lx}{\Lambda}
\newcommand{\Ox}{\Omega}
\newcommand{\Dx}{\Delta}
\newcommand{\Gx}{\Gamma}
\newcommand{\te}{\tilde \epsilon}
\newcommand{\tm}{\tilde \mu}
\newcommand{\tmu}{\tilde \mu}

%
%%%%%%%%%%%%%%%%%%%%%%%%%%%%%%%%%%%%%%%%%%
%%%%%%%%%%%%%%%%%%%%%
% Cal
\newcommand{\cC}{\mathcal{C}}
\newcommand{\cD}{\mathcal{D}}
\newcommand{\cL}{\mathcal{L}}
\newcommand{\cK}{\mathcal{K}}
\newcommand{\cN}{\mathcal{N}}
\newcommand{\cG}{\mathcal{G}}
\newcommand{\cA}{\mathcal{A}}
\newcommand{\cF}{\mathcal{F}}
\newcommand{\cH}{\mathcal{H}}
\newcommand{\cI}{\mathcal{I}}
\newcommand{\cR}{\mathcal{R}}
\newcommand{\cQ}{\mathcal{Q}}
\newcommand{\cM}{\mathcal M}
\newcommand{\W}{\mathcal W}
\newcommand{\V}{\mathcal V}
\def\Im{\,{\rm Im}\, }
\def\Re{\,{\rm Re}\, }

\newcommand{\nsub}{n_\mathrm{sub}}

%%%%%%%%%%%%%%%%%%%%%%%%%%%%%%%%%%%%%%%%%%
%%%%%%%%%

\title{Scale variations in {\tt FK} tables}
\author{Valerio Bertone, Nathan Hartland}
\begin{document}
\maketitle
%\tableofcontents
%\clearpage
\section{Scale variations in {\tt APPLgrids}}
In the {\tt APPLgrid} framework, the computation of a generic observable $\sigma$ evaluated at some variation of the factorisation ($\xi_FQ$) and
renormalisation($\xi_RQ$) scales can be broken down into four components at next-to-leading order
\be \sigma(\xi_F, \xi_R) = \left( \sigma_B + \sigma_R + \sigma_{F1} + \sigma_{F2}\right)(\xi_F, \xi_R).\ee
Firstly we have the 'basic' component, which is simply the standard {\tt APPLgrid} product evaluated at the modified scales,
\be \sigma_B(\xi_F, \xi_R) =  \left(\tilde{\alpha}_s^{p}(\xi_R^2Q^2_\tau) 
     W_{\alpha\beta,\tau}^{(0)(s)}+ \tilde{\alpha}_s^{p+1}(\xi_R^2Q^2_\tau) W_{\alpha\beta,\tau}^{(1)(s)} \right)
     F^{(s)}_{\alpha\beta}\left[f_1, f_2\right]\left(\xi_F^2{Q^2_\tau}\right), \ee
where we use the shorthand
\be \tilde{\alpha}_s(\xi_R^2Q^2_\tau) =  \frac{1}{2\pi}\alpha_s\left(\xi_R^2 {Q^2_\tau}\right), \ee
and the subprocess parton densities are constructed as
\be \label{eq:basicsubproc}
  F^{(s)}_{\alpha\beta,\tau}\left[f_1, f_2\right]\left(\xi_F^2{Q^2_\tau}\right) =\sum_{i,j} C^{(s)}_{ij} 
  f^{(1)}_i(x_{\alpha},\xi_F^2Q^2_\tau)f_j^{(2)}(x_{\beta},\xi_F^2Q^2_\tau)\,,
\ee
for the appropriate partonic subprocess defined by the $C$. Next we have a term proportional to the logarithm of $\xi_R$:
\be \sigma_R(\xi_F, \xi_R) = \tilde{\alpha}_s^{p+1}(\xi_R^2Q^2_\tau) 
      \pi  \beta_0 p \ln \xi_R^2
         \,
         W_{\alpha\beta,\tau}^{(0)(s)}
    F^{(s)}_{\alpha\beta}\left[f_1, f_2\right]\left(\xi_F^2{Q^2_\tau}\right)  \ee
and terms proportional to the logarithm of $\xi_F$:
\be \sigma_{F1}(\xi_F, \xi_R) =-\tilde{\alpha}_s^{p+1}(\xi_R^2Q^2_\tau)
\ln \xi_F^2  W_{\alpha\beta,\tau}^{(0)(s)}
F^{(s)}_{\alpha\beta}[P_0\otimes f_1, f_2](\xi_F^2{Q^2_\tau}),
\ee
\be \sigma_{F2}(\xi_F, \xi_R) =-\tilde{\alpha}_s^{p+1}(\xi_R^2Q^2_\tau)
\ln \xi_F^2  W_{\alpha\beta,\tau}^{(0)(s)}
F^{(s)}_{\alpha\beta}[f_1, P_0\otimes f_2](\xi_F^2{Q^2_\tau}).
\ee
Here the parton subprocesses are calculated using one PDF and one convolution of PDFs with leading order splitting functions $P_0$:
\be \label{eq:convsubprocess1} F^{(s)}_{\alpha\beta}[P_0\otimes f_1, f_2](\xi_F^2{Q^2_\tau})=\sum_{i,j} C^{(s)}_{ij} 
 f_j^{(2)}(x_{\beta},\xi_F^2Q^2_\tau)\,  \int_{x_\alpha}^1 \frac{dz}{z}P^0_{ik}\left(\frac{x_\alpha}{z}\right)f^{(1)}_k(z,\xi_F^2Q^2_\tau), \ee

\be \label{eq:convsubprocess2} F^{(s)}_{\alpha\beta}[f_1, P_0\otimes f_2](\xi_F^2{Q^2_\tau})=\sum_{i,j} C^{(s)}_{ij} 
 f_j^{(1)}(x_{\alpha},\xi_F^2Q^2_\tau)\,  \int_{x_\beta}^1 \frac{dz}{z}P^0_{jl}\left(\frac{x_\beta}{z}\right)f^{(2)}_l(z,\xi_F^2Q^2_\tau). \ee
%%%%%%%%%%%%%%%%%%%%%%%%%%%%%%%%%%%%%%%%%%%%%%%%%%%%%%%%%%%%%%%%%%%%%%%%%
\section{Interpolated procedure for factorisation scale variations}
In {\tt FK} tables we relate the general-scale PDFs used above to initial scale evolution basis PDFs through DGLAP evolution factors as
\begin{equation}\label{eq:fastPDFfinal_recalled}
  f_i(x_{\alpha},\xi_F^2Q^2_\tau) = \sum_{k}
  \sum_\beta A^{(\tau)}_{\alpha\beta, ik}\;
  N_k(x_\beta)\,, 
\end{equation}
where the $N$ are evolution basis PDFs evaluated at $Q_0^2$. It is therefore simple to write the subprocess densities in Eqn. \ref{eq:basicsubproc} in terms of initial state PDFs
\begin{equation}\label{eq:FK1}
\begin{array}{rcl}
F^{(s)}_{\alpha\beta}[f_1, f_2](\xi_F^2Q^2_\tau) &=&  \displaystyle \sum_{i,j} \sum_{k,l}
                               \sum_{\delta,\gamma} C^{(s)}_{ij}
                               \left[  A^{(\tau)}_{\alpha\delta ik}\;
                               N_k(x_\delta) A^{(\tau)}_{\beta\gamma
                               jl}\; N_l(x_\gamma) \right]\;\;\;
  \\
\\
&=& \displaystyle \sum_{k,l}\sum_{\delta,\gamma}
\widetilde{C}^{(s),\tau}_{kl,\alpha\beta\gamma\delta}
N_k(x_\delta) N_l(x_\gamma)\,.
\end{array}
\end{equation}
To do the same for the subprocess densities in terms of convolutions in Eqns. \ref{eq:convsubprocess1}, \ref{eq:convsubprocess2} requires some more legwork.
Firstly we recall that in this framework PDFs are represented upon an interpolation grid
\be f_i(x,Q_\tau^2) \simeq \sum_\gamma^{N_x} f_i(x_\gamma, Q_\tau^2)\mathcal{I}^{(\gamma)}(x).\ee
Therefore we can write the convolutions in Eqns. \ref{eq:convsubprocess1}, \ref{eq:convsubprocess2} as
\be \label{eq:deconvolute} \left[P_0\otimes f\right]_{i}(x_\alpha,  \xi_F^2Q_\tau^2) = \sum_\gamma \sum_k B_{\alpha\gamma, ik} f_k(x_\gamma, \xi_F^2Q_\tau^2),\ee
where
\be B_{\alpha\gamma, ik} = \int_{x_\alpha}^1 \frac{dz}{z} P_{ik}^0\left(\frac{x_\alpha}{z}\right)\mathcal{I}^{(\gamma)}(z).\ee
Therefore with Eqn. \ref{eq:deconvolute} we can place the subprocess densities involving convolutions at the same level as the standard subprocess densities, directly in terms of parton distributions evaluated upon an interpolation grid.

\be \label{eq:deconvsubprocess1} F^{(s)}_{\alpha\beta}[P_0\otimes f_1, f_2](\xi_F^2{Q^2_\tau})=\sum_\gamma \sum_{i,j,k} C^{(s)}_{ij} B_{\alpha\gamma, ik} \, 
 f_k^{(1)}(x_\gamma, \xi_F^2Q_\tau^2)\, f_j^{(2)}(x_{\beta},\xi_F^2Q^2_\tau), \ee
\be \label{eq:deconvsubprocess1} F^{(s)}_{\alpha\beta}[f_1, P_0\otimes f_2](\xi_F^2{Q^2_\tau})=\sum_\gamma \sum_{i,j,k} C^{(s)}_{ij} B_{\beta\gamma, jk} \, 
 f_i^{(1)}(x_\alpha, \xi_F^2Q_\tau^2)\, f_k^{(2)}(x_{\gamma},\xi_F^2Q^2_\tau). \ee
 These expressions are now more naturally amenable to the precomputation of their evolution as per Eqn. \ref{eq:FK1}. The terms proportional to logarithms of $\xi_F$ are therefore (sums implied)
 \begin{align}
\sigma_{F1} &=
-\tilde{\alpha}_s^{p+1}(\xi_R^2Q^2_\tau)
\ln \xi_F^2  
W_{\alpha\beta,\tau}^{(0)(s)} C^{(s)}_{ij} B_{\alpha\gamma, ik} \, 
 f_k^{(1)}(x_\gamma, \xi_F^2Q_\tau^2)\, f_j^{(2)}(x_{\beta},\xi_F^2Q^2_\tau) \nonumber \\
 &=
 -\tilde{\alpha}_s^{p+1}(\xi_R^2Q^2_\tau)
\ln \xi_F^2  
W_{\alpha\beta,\tau}^{(0)(s)} C^{(s)}_{ij} B_{\alpha\gamma, ik} \, 
A^{(\tau)}_{\gamma\delta, km} A^{(\tau)}_{\beta\epsilon, jn} N^{(1)}_{m}(x_\delta) N^{(2)}_{n}(x_\epsilon),
 \end{align}
and
\begin{align}
\sigma_{F2} &=
-\tilde{\alpha}_s^{p+1}(\xi_R^2Q^2_\tau) \ln \xi_F^2  
W_{\alpha\beta,\tau}^{(0)(s)} C^{(s)}_{ij} B_{\beta\gamma, jk} \, 
 f_i^{(1)}(x_\alpha, \xi_F^2Q_\tau^2)\, f_k^{(2)}(x_{\gamma},\xi_F^2Q^2_\tau) \nonumber \\
 &=
-\tilde{\alpha}_s^{p+1}(\xi_R^2Q^2_\tau) \ln \xi_F^2  
W_{\alpha\beta,\tau}^{(0)(s)} C^{(s)}_{ij} B_{\beta\gamma, jk} \, 
A^{(\tau)}_{\alpha\delta, im} A^{(\tau)}_{\gamma\epsilon, kn} N^{(1)}_{m}(x_\delta) N^{(2)}_{n}(x_\epsilon).
\end{align}
It is then simple to arrange a total term proportional to $\log \xi_F$:
\begin{align}
\sigma_F &= \sigma_{F1} + \sigma_{F2}\\
&= -\tilde{\alpha}_s^{p+1}(\xi_R^2Q^2_\tau) \ln \xi_F^2  W_{\alpha\beta,\tau}^{(0)(s)} C^{(s)}_{ij} N^{(1)}_{m}(x_\delta) N^{(2)}_{n}(x_\epsilon)\\
&\times \left\{ B_{\alpha\gamma, ik} \, 
A^{(\tau)}_{\gamma\delta, km} A^{(\tau)}_{\beta\epsilon, jn} +  B_{\beta\gamma, jk} \, 
A^{(\tau)}_{\alpha\delta, im} A^{(\tau)}_{\gamma\epsilon, kn}  \right\}.
\end{align}

\end{document} 
